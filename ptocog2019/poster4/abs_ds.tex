\documentclass[11pt]{article}
%Gummi|065|=)
\title{\textbf{Performance of Prompt Gamma fall-off detection in clinical simulations}}
\author{Brent F. B. Huisman}
\date{}
\begin{document}


\maketitle


As the first Prompt Gammas (PG) cameras are deployed in clinical setting, we studied PG fall-off positions (FOP) estimation on a complete clinical simulations. The number of protons (spot weight) required for a consistent FOP estimate was investigated for two PG cameras, a multislit and knife edge design, for a single spot of a clinical fully clinical simulation of an patient treatment. A new spot-grouping method is proposed that combines better measurement statistics with fall-off preservation.

We considered a clinical head and neck treatment plan containing both a CT and re-planning (RP)CT. Monte-Carlo simulations were performed on both CTs. During the irradiation, two PG cameras implemented as published, recorded the PG profiles, spot by spot. We study shifted distributions as function of the spot weights, for each camera, from $10^6-10^9$. A FOP estimation was applied on 20 CT and 20 RPCT realizations to obtain FOP distributions for both images. A natural way to improve statistics is to integrate PG profiles over multiple spots. We investigate if and how spot-grouping methods improve FOP estimation.

By studying recent treatment plans from various proton clinics, we observe very few spots with weights over $10^8$. We did not manage to detect the morphological change present, an approximately 10mm shift, between the (RP)CT with either PG camera, with such statistics. Only for $10^9$ primaries, with one of the cameras, the change may be expected to be detected. Preliminary results of grouping spots in geometric layers seem to make an improved FOP estimation possible.

\end{document}

