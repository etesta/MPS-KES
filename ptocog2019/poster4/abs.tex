\documentclass[11pt]{article}
%Gummi|065|=)
\title{\textbf{Performance of Prompt Gamma fall-off detection in clinical simulations}}
\author{Brent F. B. Huisman}
\date{}
\begin{document}

\maketitle


As the first Prompt Gammas (PG) cameras are deployed in clinical setting, we present a feasibility study of PG fall-off position (FOP) estimation on fully clinical simulations. The statistics required for a consistent FOP estimate is investigated for two PG detectors. A new spot-grouping method is proposed that combines better measurement statistics with fall-off preservation.

Various proton clinics donated proton TPs for a study of typical spot weight distributions. A head and neck case with a CT and replanning (RP)CT was selected. A proton TP was constructed on the CT and irradiated on both CT and RPCT. During the irradiation, two PG cameras implemented according to their published specifications, recorded the PG profile. A FOP estimation algorithm is presented, and executed on 20 CT and 20 RPCT realizations to obtain a shift distribution. A selected spot has its weight modulated, in order to study shift distributions as function of the spot weight, for each camera. Then, a new dose-range based spot-grouping method is presented and evaluated.

Spot weight averages are specific for each condition, and for cases where treatment verification is most pertinent, an average of order $10^7$ is found. Neither PG camera produces reliable FOP estimates with such statistics, even though the selected case exhibits significant morphological change. Only for $10^9$ primaries, with one of the cameras, the change may be expected to be detected. When spots are grouped, our method produces slightly better FOP estimation than energy-layer grouping.

\end{document}
